% LTeX: enabled=false
% Language settings ========================================
\usepackage[ngerman]{babel}				% language support for in-text language (e.g. auto-hyphenation)
% alternative: [ngerman]


% Encoding, charset, textformatting ========================
\usepackage[utf8]{inputenc}				% include utf8 charset - special characters, load before csquotes!
\usepackage{csquotes} 					% add quotations in text
\usepackage[T1]{fontenc}				% defines charset to west european chars and improves word breaks
\usepackage{lmodern}					% change font from standard to lmodern
\usepackage[normalem]{ulem}						% underline text with \uline \uuline; \normalem before bibliography
\useunder{\uline}{\ul}{}
\usepackage[usenames, table, dvipsnames]{xcolor}	% adds other font colors, table enables coloring of tables
\usepackage{pdflscape}					% make landscape Pages using \begin{landscape} and \end{landscape}
\usepackage{blindtext}                  % german "lorem ipsum" text (http://tex.lickert.net/packages/blindtext/)
%\usepackage{fontawesome}				% includes web-related icons, optional
%\usepackage{soul}						% adds further text highlighting options, optional

% Page Layout ==============================================
\usepackage{parskip}					% implements paragraph layouts when vertical space is used
\usepackage{geometry}					% define borders, layout and optional binding-offset
\geometry{
	left = 30mm,						% with offset 10mm: 30mm // without: 35mm
	right = 30mm,						% with offset 10mm: 30mm // without: 35mm
	top = 35mm,
	bottom = 35mm,
	bindingoffset = 10mm				% recommended offset for printing: 10mm
}


% Custom footer and header =================================
\usepackage{fancyhdr}
\pagestyle{fancyplain}
\renewcommand{\headrulewidth}{0.3pt} 	
\renewcommand{\plainheadrulewidth}{0.3pt}	% optional horizontal line at top
\renewcommand{\footrulewidth}{0.3pt} 	
\renewcommand{\plainfootrulewidth}{0.3pt}	% optional horizontal line at bottom
\fancyhf{}
\lhead{}
\chead{}
\rhead{}
\lfoot{}
\cfoot{\thepage}
\rfoot{}
\setlength{\headheight}{14.5pt}


% Graphics and tables ======================================
\usepackage{graphicx}			% package for \includegraphics key-value parameters, include before wrapfig
\usepackage{float}				% provides more strict figure positioning than default - mode h!
\usepackage{wrapfig}			% enables graphics and tables to be wrapped by text
\usepackage{subcaption}			% add captions to subfigures // can be used instead of subfigure
\usepackage{caption}			% package to customize captions, see next line
\newcommand{\source}[1]{\caption*{Source: {#1}} }
\usepackage{tabularx}			% define table using ascii-art, optional
\usepackage{multirow}
\usepackage{array}
\newcolumntype{M}[1]{>{\centering\arraybackslash}m{#1}} % M: horizontal and vertical entered with custom width
\newcolumntype{L}[1]{>{\raggedright\arraybackslash}p{#1}} % L: left aligned with custom width
\newcolumntype{C}[1]{>{\centering\arraybackslash}p{#1}} % C: horizontal centered with custom width
\newcolumntype{R}[1]{>{\raggedleft\arraybackslash}p{#1}} % R: right aligned with custom width


% Source code listing ======================================
% https://tex.stackexchange.com/questions/89574/language-option-supported-in-listings
\usepackage{listings}
\renewcommand{\lstlistingname}{Source code}
\renewcommand{\lstlistlistingname}{Quellcodeverzeichnis}

% Language highlighting, optional ==========================
% requires package xcolor
\usepackage[outputdir=Build]{minted}
\setminted[]{
    frame=lines,
    framesep=2mm,
    baselinestretch=1.2,
    fontsize=\footnotesize,
    linenos,
    breaklines
}
\DeclareUnicodeCharacter{21FE}{$\rightarrow$}
\newenvironment{code}{\captionsetup{type=listing}\list{}{%
\leftmargin=0em
\topsep=2ex
\parsep=\parskip
\listparindent=\parindent
\itemindent=\parindent
}\item\relax}
{\endlist}

\usepackage{pdfpages}
% Use \includepdf[pages=-]{myfile.pdf} to include all pages of an PDF file

% Bib source and backend ===================================
\usepackage[		% includes package for bibliography with biblatex
backend = biber,
autocite = inline,	% cite automatically in text, default: inline; alternatives: footnote
style = ieee,		% citation styles, default: ieee; alternatives: authoryear, numeric, verbose, alphabetic
urldate = comp,
dashed = false
]{biblatex}
\usepackage{xpatch}
\usepackage{url}
% in case of problems with bibliography:
% check if editor is configured to compile bibliography with biber as defined in backend
% check for package updates of biblatex and biber in case error meassage reports incompatibility
% styles of bibliography can be extended with the package natbib


% Text structure ===========================================
\usepackage[hyperfootnotes=false, hidelinks]{hyperref} 	% package for hyperlinks and additional PDF bookmarks
\hypersetup{
    colorlinks=true,
    linkcolor=blue,
    filecolor=magenta,      
    urlcolor=cyan,
    citecolor=RoyalPurple
    }
\usepackage{nameref}							% package to referencing sections by name instead of number
\setlength{\marginparwidth}{2.5cm}				% defines width of comments / todos in margins
\usepackage{todonotes}							% includes notes for todos and reminder
\usepackage[toc, page]{appendix}				% package for appendices
\renewcommand{\appendixpagename}{Appendix}
\renewcommand{\appendixtocname}{Appendix}
%\usepackage[toc, automake]{glossaries}			% package for glossary, load after hyperref
\interfootnotelinepenalty=10000 % Dont break footnotes onto multiple pages
\usepackage{setspace} 	% used for line spacing, recommended instead of \linespread or \baselinestretch
\onehalfspacing			% write on 1.5 lines, for 1 line use \singlespacing, for 2 lines \doublespacing


% Acronyms
\usepackage[printonlyused]{acronym}	% package for defining and referencing acronyms

% Custom variables and values ==============================
\newcommand{\confTitle}[1]{\def\varTitle{#1}\hypersetup{pdftitle={#1}}}
\newcommand{\confAuthor}[1]{\def\varAuthor{#1}\hypersetup{pdfauthor={#1}}}
\newcommand{\confAuthorA}[1]{\def\varAuthorA{#1}}
\newcommand{\confAuthorB}[1]{\def\varAuthorB{#1}}
\newcommand{\confAuthorC}[1]{\def\varAuthorC{#1}}
\newcommand{\confLocation}[1]{\def\varLocation{#1}}
\newcommand{\confReportType}[1]{\def\varReportType{#1}}
\newcommand{\confDate}[1]{\def\varDate{#1}}
\newcommand{\confCompany}[1]{\def\varCompany{#1}}

\confTitle{Entwicklung einer KI für das Schach-Spiel}
\confAuthorA{Philipp Fruck}
\confAuthorB{Anton Plagemann}
\confAuthorC{Pius Walter}
\confAuthor{\varAuthorA, \varAuthorB, \varAuthorC}
\confLocation{Mannheim}
\confReportType{T3\_3200}
\confDate{19. April 2022}
\confCompany{Hewlett-Packard GmbH \&\\&Schwarz Dienstleistung KG}
